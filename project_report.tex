\documentclass[a4paper]{article}

\usepackage[per-mode=symbol,separate-uncertainty=true]{siunitx}
\usepackage{amsmath}
\usepackage{float}
\usepackage{graphicx}
\usepackage[a4paper,top=3cm,bottom=2cm,left=3cm,right=3cm,marginparwidth=1.75cm]{geometry}
\usepackage{mathtools}
\usepackage{subcaption}
\usepackage{xcolor}

\title{Digital Microelectronics 2018 \\ Final project report}
\author{Group 5}

\begin{document}
\maketitle

\section{Introduction}

\section{Inverter}

\section{Half adder}

\subsection{Layout}
Given that the layout of the half adder is quite more complex than the one of the inverter, we started by deeply analyzing the possible approaches using the pen-and-paper method. The final solution that we found consisted in sharing source/drain diffusions as much as possible, in order to minimize the area in the horizontal direction.

When actually drawing the design in Virtuoso, we made sure of running the Design Rule Checker very often, in order to avoid problems in advance. 

\end{document}